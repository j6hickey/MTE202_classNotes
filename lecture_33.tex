
\chapter*{Lecture 33}

\begin{recall}{}{}
\begin{itemize}
\item Method of partial fraction 
(quadratic factors)
\end{itemize}
\end{recall}





\textbf{last lecture}

How did we do this jump?
\begin{equation*}
F(s)=\frac{\frac{s}{2}}{s^2+s+\frac{3}{4}}=\frac{E(s+\frac{1}{2}) + D\frac{\sqrt{2}}{2}}{(s+\frac{1}{2})^2+(\frac{\sqrt{2}}{2})^2}
\end{equation*}

We recall that we want to obtain:
\begin{align*}
&\Lapl^{-1}\left[\frac{s-a}{(s-a)^2+b^2}\right]=e^{at}\cos(bt)\\
&\qquad \text{and}\\
&\Lapl^{-1}\left[\frac{b}{(s-a)^2+b^2}\right]=e^{at}\sin(bt)
\end{align*}




\subsection{Translation in s}
If the Laplace transform $\Lapl (f)=F(s)$ exists for $s>\alpha$, then:
\begin{equation}
\Lapl (e^{at}f)= F(s-a)
\end{equation}
for $s>\alpha+a$.

This property can be proven by computing the Laplacian:
\begin{align*}
\Lapl (e^{at}f)&= \int^\infty_0 e^{-st}e^{at}f(t) dt\\
&= \int^\infty_0 e^{(a-s)t}f(t) dt\\
&=F(s-a)
\end{align*}



\subsection{Convolution}
Let $f(t)$ and $g(t)$ satisfy the existance theorem. Then, the product of their transform $F(s)=\Lapl(f)$ amd $G(s)=\Lapl(g)$ is the transform $H(s)=\Lapl(h)$ of the \textbf{convolution} of $f(t)$ and $g(t)$:
\begin{align*}
h(t) = (f*g)(t) = \int^t_0 f(\tau)g(t-\tau) d\tau
\end{align*}


\begin{exmp}{Convolution}{}
Find the inverse of:
\begin{align*}
H(s)=\frac{1}{(s^2+1)^2}=\frac{1}{s^2+1}\cdot \frac{1}{s^2+1}
\end{align*}
\textbf{Solution}\\
We know that each factor on the right has the inverse $\sin t$. Therefore:\begin{align*}
h(t)&=\Lapl^{-1}(H)=\sin t * \sin t  \\
&=\int^t_0 \sin \tau \sin (t-\tau)d\tau\\
\end{align*}
Trigonometric identify:
\begin{align*}
\sin x\sin y = \frac{1}{2}\left[\cos(x-y)-\cos(x+y)\right]
\end{align*}
We write:
\begin{align*}
\sin \underbrace{\tau}_x \sin \underbrace{(t-\tau)}_y =  \frac{1}{2}\left[\cos(2\tau-t)-\cos(t)\right]
\end{align*}

\begin{align*}
h(t)&=\Lapl^{-1}(H)=\sin t * \sin t\\
&=\int^t_0 \sin \tau \sin (t-\tau)d\tau\\
&= -\frac{1}{2}t\cos t+\frac{1}{2}\sin t
\end{align*}

\end{exmp}



\subsection{Treatment of discontinuous functions}
Convert discontinuous functions expressed over individual subdomains into a single function using the Heaviside step function.

\begin{equation*}
H(t)=
\begin{cases}
0 \qquad \qquad &t<0\\
1 &t\geq 0
\end{cases}
\end{equation*}

Alternatively (using translation):
\begin{equation*}
H(t-a)=
\begin{cases}
0 \qquad \qquad &t<a\\
1 &t\geq a
\end{cases}
\end{equation*}
Or by modifying the amplitude:

Alternatively (using translation):
\begin{equation*}
MH(t-a)=
\begin{cases}
0 \qquad \qquad &t<a\\
M &t \geq a
\end{cases}
\end{equation*}

$H(t)$ is used as a building block to convert piecewise continuous functions into a single function.

\begin{exmp}{}
Use the Heaviside function to transform:
\begin{equation*}
y(t)=
\begin{cases}
0 \qquad \qquad &t\leq 1\\
5 &1<t \leq2\\
0&t>2
\end{cases}
\end{equation*}
\textbf{Solution:}\\
Let: $y_1(t)=5H(t-1)$ and $y_2=5H(t-2)$, then: $y(t)=y_1(t)-y_2(t)=5\left[H(t-1)-H(t-2)\right]$
\end{exmp}



\begin{exmp}{}
Use the Heaviside function to transform:
\begin{equation*}
f(t)=
\begin{cases}
2+t^2 \qquad \qquad &t\leq 2\\
6 &2 < t\leq 3\\
\frac{2}{2t-5}&3<t< \infty
\end{cases}
\end{equation*}
\textbf{Solution:}\\
\begin{align*}
f(t)&=(2+t^2)\left[H(t-0)-H(t-2)\right]\\
 &+6\left[H(t-2)-H(t-3)\right]\\
 &+\frac{2}{2t-5}\left[H(t-3)\right]
\end{align*}

\end{exmp}


