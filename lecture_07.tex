\chapter*{Lecture 7}
\begin{recall}{}{}
\begin{itemize}
\item integrating factors
\end{itemize}
\end{recall}









\begin{exmp}{Integrating factors:}\\
Solve:
\begin{equation*}
(2\sin(y^2))dx+xy\cos(y^2)dy=0
\end{equation*}
With the initial condition $y(2)=\sqrt{\pi/2}$

\textbf{Solution:}
\begin{enumerate}
\item Check for exactness: 
\begin{equation*}
P=2 \sin(y^2) \qquad \text{and} \qquad Q=xy\cos(y^2)
\end{equation*}
and
\begin{equation*}
\frac{\partial P}{\partial y}=4y\cos(y^2)\qquad \text{and} \qquad \frac{\partial Q}{\partial x}=y\cos(y^2)
\end{equation*}
the equation is NOT exact!
\item Find the integrating factors (assume $U=U(x)$):
\begin{equation*}
R=\frac{1}{Q}\left(\partial P/\partial y-\partial Q/\partial x\right)=\frac{1}{xy\cos(y^2)}\left(4y\cos(y^2)-y\cos(y^2)\right)=\frac{3y}{xy}=\frac{3}{x}
\end{equation*}
Therefore:


\begin{equation*}
U(x)=\exp \int R(x)dx=\exp{\int\frac{3}{x}dx} ={x}^3
\end{equation*}
(recall: $e^{3\ln\left|x\right|}=x^3$) Our integrating factor is ${x}^3$!

We can re-write the original equation as an exact equation!
\begin{equation*}
{x}^3(2\sin(y^2))dx+{x}^4y\cos(y^2)dy=0
\end{equation*}
(verify, the exactness!)

\item Solve the exact equation:
\begin{equation*}
F(x,y)=\int M dx=  \int (2{x}^3\sin(y^2)) dx+ g(y) =  \frac{1}{2}{x}^4\sin(y^2)) +g(y)
\end{equation*}
we derive with respect to $y$ in order to find the term $g(y)$
\begin{equation*}
\frac{\partial F(x,y)}{\partial y}= {x}^4y\cos(y^2) +g'(y) = \underbrace{{x}^4y\cos(y^2)}_{N(x,y)}
\end{equation*}
Hence:  $g'(y)=0$ and $g(y)=cst$.
The general solution is then:
\begin{equation*}
F(x,y)= \frac{1}{2}{x}^4\sin(y^2) +c
\end{equation*}
\item Apply IC: $y(2)=\sqrt{\pi/2}$

\begin{equation*}
c= \frac{1}{2}{x}^4\sin(y^2) \qquad \text{or}\qquad c= \frac{1}{2}{2}^4\sin(\sqrt{\pi/2})
\end{equation*}

Particular solution is then:
\begin{equation*}
\boxed{{x}^4\sin(y^2) =16}
\end{equation*}
\end{enumerate}
\end{exmp}

\begin{center}
\noindent\rule{4cm}{0.4pt}
\end{center}


So far, we have assumed that the integrating factor $U$ is a function of $x$. In other words: $U=U(x)$. But we could also consider a case where $U$ depends only on $y$: $U=U(y)$. What is the difference?
\begin{itemize}
\item Suppose we assume $U=U(x)$ 
\begin{equation*}
\cancel{P(x,y)\frac{\partial U(x)}{\partial y}}+U(x)\frac{\partial P(x,y)}{\partial y}={Q(x,y)\frac{\partial U(x)}{dx}}+U(x)\frac{\partial Q(x,y)}{\partial x}
\end{equation*}
After manipulations (recall equation \eqref{chainrule}), we obtain:
\begin{equation*}
\boxed{U=\exp\left[\int \frac{1}{Q}\left(\frac{\partial P}{\partial y}-\frac{\partial Q}{\partial x}\right)  dx\right]}
\end{equation*}

\item Now suppose we set $U=U(y)$
\begin{equation*}
P(x,y)\frac{\partial U(y)}{\partial y}+U(y)\frac{\partial P(x,y)}{\partial y}=\cancel{Q(x,y)\frac{\partial U(y)}{dx}}+U(y)\frac{\partial Q(x,y)}{\partial x}
\end{equation*}
A similar manipulation leads to:
\begin{equation*}
\boxed{U=\exp\left[\int \frac{1}{P}\left(\frac{\partial Q}{\partial x}-\frac{\partial P}{\partial y}\right) dy\right]}
\end{equation*}
\end{itemize}

How to select between $U(x)$ and $U(y)$?
\begin{itemize}
\item If $R=(\frac{\partial P}{\partial y}-\frac{\partial Q}{\partial x})/Q$ is \textbf{continuous} and  \textbf{depends only on} $x$, then: \\{\large - select $U(x)$}
\item If $R=(\frac{\partial Q}{\partial x}-\frac{\partial P}{\partial y})/P$ is \textbf{continuous} and \textbf{depends only on} $y$, then: \\{\large - select $U(y)$}
\end{itemize}



\updateinfo[September 25, 2018]