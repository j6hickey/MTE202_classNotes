
\chapter*{Lecture 29}

\begin{recall}{}{}
\begin{itemize}
\item \url{http://evaluate.uwaterloo.ca/}
\item Course evaluation today at about 2pm.
\item Existance of LT
\begin{itemize}
\item Pieceqise continuous
\item Exponential order
\end{itemize}
\item Linearity of the transform
\end{itemize}
\end{recall}

\subsection*{Recalls:}
\begin{itemize}
\item \textbf{Exponential order remark:} if $f(t)$ is of exponential order, we said:
\begin{equation*}
\left|f(t)\right|\leq M e^{\alpha t} \qquad \qquad \text{for } t>T
\end{equation*}
By multiplying both sides by  $e^{-\alpha t}$, the above equation is equivalent to:
\begin{equation*}
e^{-\alpha t}\left|f(t)\right|\leq M \qquad \qquad \text{for } t>T
\end{equation*}
In other words, we are saying that the product $e^{-\alpha t} \left|f(t)\right|$ is bounded (corresponds to our integrand in the Laplace transform) for the range of $t$ of interest.

If $f(t)=e^{t^2}$, find the exponential order?

\begin{equation*}
e^{-\alpha t}\left|f(t) \right|=e^{-\alpha t}e^{t^2}=e^{t^2-\alpha t} \leq ?  \end{equation*}

when $t\rightarrow \infty$ the LHS also goes to $\infty$. Therefore $e^{t^2}$ is not of exponential order!

\item \textbf{Improper integrals:} It is important to recall that we are dealing with {improper integrals}. Therefore, for exactness, we should be solving:
\begin{equation}
\int^\infty_0 e^{-st} f(t) dt=\lim_{n \rightarrow \infty} \int^N_0 e^{-st} f(t) dt
\end{equation}
\end{itemize}


\subsection{Translation in s}
If the Laplace transform $\Lapl (f)=F(s)$ exists for $s>\alpha$, then:
\begin{equation}
\Lapl (e^{at}f)= F(s-a)
\end{equation}
for $s>\alpha+a$.

This property can be proven by computing the Laplacian:
\begin{align*}
\Lapl (e^{at}f)&= \int^\infty_0 e^{-st}e^{at}f(t) dt\\
&= \int^\infty_0 e^{(a-s)t}f(t) dt\\
&=F(s-a)
\end{align*}


\subsection{Laplace transform of derivatives}
The usefulness of LT will soon become apparent!
\begin{align*}
\Lapl (f')&= \underbrace{\int^\infty_0 e^{-st}\frac{df(t)}{dt} dt}_{\text{integration by parts}}\\
\end{align*}
We define an integration by parts with $u=e^{-st}$ and $dv=f'dt$
\begin{align*}
\Lapl (f')&= \left. \underbrace{e^{-st}f(t)}_\text{need exp. order cond}\right|^\infty_0+\left. s\int^\infty_0e^{-st}f(t) dt\right|^\infty_0\\
&=0 -f(0) + s\underbrace{\left. \int^\infty_0e^{-st}f(t) dt\right|^\infty_0}_{\Lapl(f)}
\end{align*}
Therefore:
\begin{align*}
\boxed{\Lapl \left(f'(t)\right)= s\Lapl(f)-f(0)}
\end{align*}
where $f(0)$ is the initial condition of the problem!


Similarly:
\begin{align*}
\boxed{\Lapl \left(f''(t)\right)= s^2F(s)-s f(0)-f'(0)}
\end{align*}
[I invite you to derive this last equation by yourself at home]

\begin{exmp}{LT ODE :}\\
Consider the following ODE:
\begin{align*}
f''+2f'+f=0 \qquad \qquad \text{with } &f(0)=1; \\
&f'(0)=0
\end{align*}
Find $\Lapl(f'')$.

\textbf{Solution}\\
Take the LT of both sides:
\begin{align*}
\qquad & \Lapl(f''+2f'+f)=\Lapl(0)\\
&=\Lapl(f'')+2\Lapl(f')+\Lapl(f) \qquad \text{(principle of linearity)}\\
&=\underbrace{s^2 F(s) - sf(0)-f'(0)} +2\left(\underbrace{sF(s) - f(0)}\right) +F(s)\\
&(s^2+2s+1)F(s)-(s+2)f(0)-f'(0)=(s^2+2s+1)F(s)-(s+2)1-0\\
\end{align*}
Therefore:
\begin{align*}
F(s)=\frac{s+2}{(s+1)^2}
\end{align*}

Now, we can find $\Lapl(f'')$:
\begin{align*}
\Lapl \left(f''\right)&= s^2F(s)-s f(0)-f'(0) = s^2 \frac{s+2}{(s+1)^2}-s(1)-0\\
&s^2\frac{s+2}{(s+1)^2} -s=\\
&\boxed{\frac{-s}{(s+1)^2}}
\end{align*}
\end{exmp}
