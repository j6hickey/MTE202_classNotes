\chapter*{Lecture 16}
\begin{recall}{}{}
\begin{itemize}
\item Reduction of order technique
\begin{itemize}
\item missing terms
\item find second solution
\end{itemize}
\item Auxiliary equations
\end{itemize}
\end{recall}




\section{Auxiliary equation} 
This is the second technique that we will study. This approach works for \textbf{constant coefficient, homogeneous ODE}.

Consider an ODE in general form:
\begin{equation}
\boxed{a\frac{d^2 y}{dx^2}+b\frac{d y}{dx}+cy=0}
\label{genForm} 
\end{equation}

For the special case when $a=1$, $b=0$ and $c=-1$:
\begin{equation*}
\frac{d^2 y}{dx^2}=y
\end{equation*}
The solutions are: $y_1=e^x$ and $y_2=e^{-x}$. We note that both solutions are fairly similar.

This observation leads us to making an assumption that for constant coefficient, homogeneous ODE, the solution to \eqref{genForm} is: $y=e^{rx}$. Given this assumption, $y'=re^{rx}$ and $y''=r^2e^{rx}$.

If we sub into \eqref{genForm}, the solution is:
\begin{equation*}
ar^2e^{rx}+b r e^{rx} +ce^{rx}=0
\end{equation*}
By simplifying (this equation must be homogeneous for this to work!), we obtain:
\begin{equation}
\boxed{ar^2+b r  +c=0}
\label{auxi}
\end{equation}
Therefore, if we find $r$ that satisfies the above equation, we can find the solution(s) $y=e^{rx}$ that satisfies the ODE.

Equation \eqref{auxi} is called the \textbf{auxiliary equation}.

We recall the general form of the root equation is:
\begin{equation*}
r=\frac{-b\pm \sqrt{b^2-4ac}}{2a}
\end{equation*}

Three possible cases exist for the roots of equation \eqref{auxi}:
\begin{itemize}
\item two real, distinct roots: $b^2-4ac>0$
\item two complex, distinct roots: $b^2-4ac<0$
\item two real, equal roots: $b^2-4ac=0$
\end{itemize}

We will study each possible case:
\subsection{Two real, distinct roots: $b^2-4ac>0$}
In this case, the solutions take the form of:
\begin{equation*}
y_1=e^{r_1x} \qquad \qquad y_2=e^{r_2x}
\end{equation*}
We can show that these solution are linearly independent by looking at the Wronskian:
\begin{equation*}
W(y_1,y_2)=y_1y'_2-y_2y'_1=r_2 e^{r_1x}e^{r_2x}-r_1 e^{r_1x}e^{r_2x}=(r_2-r_1) e^{r_1x}e^{r_2x}
\end{equation*}
Therefore, as long at $r_1\neq r_2$, the equations are linearly independent.

The general solution takes the form of:
\begin{equation*}
\boxed{y_{general}=c_1y_1+c_2y_2=c_1e^{r_1x}+c_2e^{r_2x}}
\end{equation*}

\begin{center}
\noindent\rule{4cm}{0.4pt}
\end{center}

\begin{exmp}{Auxiliary equation (two real, distinct roots):}\\
Solve:
\begin{equation*}
2y''-3y'+y=0
\end{equation*}
\textbf{Solution:}\\
The equation has constant coefficients and is homogeneous, therefore we can solve with the auxiliary equation:
\begin{equation*}
2r^2-3r+1=0
\end{equation*}
The roots are $r_1=1$ and $r_2=1/2$.

The solutions are $y_1=e^{x}$ and $y_2=e^{x/2}$, and the general solution is:

\begin{equation*}
y_{general}=c_1e^{x}+c_2e^{x/2}
\end{equation*}
\end{exmp}
%
%\begin{exmp}{Auxiliary equation (two imaginary, distinct roots):}\\
%Solve:
%\begin{equation}
%9y''+6y'+4y=0
%\end{equation}
%Solution: \\
%We have a constant coefficient homogeneous equation therefore, we can solve with the auxiliary equation:
%\begin{equation}
%9r^2+6r+4=0
%\end{equation}
%such that:
%\begin{equation}
%\alpha = -b/2a=-1/3 \qquad \beta=\frac{1}{2a}\sqrt{4ac-b^2}=\frac{1}{\sqrt{3}}
%\end{equation}
%
%The linearly independent solutions are:
%\begin{equation}
%y_1=e^{\alpha x}\cos(\beta x) \qquad y_2=e^{\alpha x}\sin(\beta x)
%\end{equation}
%The general solution:
%
%\begin{equation}
%y_{general}=e^{-x/3}\left[c_1 \cos(\frac{x}{\sqrt{3}})+c_2 \sin(\frac{x}{\sqrt{3}})\right]
%\end{equation}
%\end{exmp}

\begin{center}
\noindent\rule{4cm}{0.4pt}
\end{center}


\subsection{Two complex, distinct roots: $b^2-4ac<0$}
In this case, our root equation can be written as :

\begin{equation*}
r=-\frac{b}{2a}\pm \frac{1}{2a}\sqrt{b^2-4ac}=\underbrace{-\frac{b}{2a}}_{\alpha}\pm \underbrace{\frac{1}{2a}\textcolor{red}{\sqrt{4ac-b^2}}}_\beta \underbrace{i}_{\sqrt{-1}}=\alpha\pm\beta i
\end{equation*}

The resulting solutions take the form:
\begin{equation*}
e^{r_1x}=e^{(\alpha+\beta i)x} \qquad e^{r_2x}=e^{(\alpha-\beta i)x}
\end{equation*}

We recall the Euler formula ("the most remarkable formula in mathematics", Feynman)
\begin{equation}
e^{ix}=\cos(x)+i \sin(x)
\end{equation}
Using this formula, we find that our solutions become:
\begin{align*}
e^{(\alpha+\beta i)x}=e^{\alpha x}\left(\cos(\beta x)+i\sin(\beta x)\right)\\
e^{(\alpha-\beta i)x}=e^{\alpha x}\left(\cos(\beta x)-i\sin(\beta x)\right)
\end{align*}
We now add together and divide by 2 these solutions to obtain $y_1$:
\begin{equation*}
y_1=0.5(e^{\alpha x}\left(\cos(\beta x)+i\sin(\beta x)\right)+e^{\alpha x}\left(\cos(\beta x)-i\sin(\beta x)\right))=e^{\alpha x}\cos(\beta x)
\end{equation*}

Subtract together the same two equations and divide by $2i$ these solutions to obtain $y_2$:
\begin{equation*}
y_2=\frac{1}{2i}(e^{\alpha x}\left(\cos(\beta x)+i\sin(\beta x)\right)-e^{\alpha x}\left(\cos(\beta x)-i\sin(\beta x)\right))=e^{\alpha x}\sin(\beta x)
\end{equation*}

Therefore, the general solution takes the form:
\begin{equation}
\boxed{y_{general}=e^{\alpha x}\left(c_1\cos(\beta x)+c_2\sin(\beta x)\right)}
\end{equation}



